\documentclass[notes]{subfile}

\begin{document}

\section{Feb. 26, 2018}

\subsection{Order of Zeros}

\begin{theorem}
    Assume $f$ is analytic in region $U$ and fix $a \in U$ such that
    \[ 0 = f(a) = f'(a) = \cdots = f^{(n)}(a) = \cdots \]
    Then $f(z) \equiv 0$ in some open disk $D_r(a)$ with
    $r > 0$.
\end{theorem}

\begin{proof}
    By our discussion of Taylor Series, there exists $r > 0$
    such that for all $z \in D_r(a)$,
    \[ f(z) = \sum_{k=0}^{\infty} \frac{f^{(k)}(a)}{k!}(z-a)^k
    = 0. \]
    Since each of the derivatives at $a$ are $0$, $f \equiv 0$ 
    in $D_r(a)$.
\end{proof}

We can do better though.

\begin{theorem}
    If there exists a point where all of $f$'s derivatives 
    vanish, then $f$ is identically $0$ on all of $U$.
\end{theorem}

\begin{proof}
    Let
    \begin{align*}
        E_1 &= \{ z \in U \, | \, 0 = f(z) = f'(z) = \cdots \}, \\
        E_2 &= \{ z \in U \, | \, f(z) \ne 0 \text{\ or\ }
        f'(z) \ne 0 \text{\ or\ } \cdots \}.
    \end{align*}
    Note that $E_1$ and $E_2$ are open and $E_1 \cap E_2 = \emptyset$.

    \noindent
    Let's now prove that $E_1$ and $E_2$ are both open.
    $E_1$ is closed because of our previous theorem.
    $E_2$ is closed because $f$ is continuous, so we can find
    a disk around $z$ such that $f^{(k)}(z) \ne 0$.

    \noindent
    Since $U$ is connected and $E_1$ and $E_2$ are open and disjoint,
    we claim that either $E_1 = \emptyset$ and $E_2 = \emptyset$.

    \begin{lemma}
        Suppose $E_1$ and $E_2$ are open sets such that they are
        open, disjoint, and their union is an open connected
        set $U$.
        
        \noindent
        Then either $E_1 = \emptyset$ and $E_2 = \emptyset$.
    \end{lemma}

    \begin{proof}
        Suppose for sake of contradiction that $E_1$
        and $E_2$ are nonempty.
        Choose $p \in E_1$ and $q \in E_2$ such that
        Since $U$ is connected, there exists a path
        $\gamma : [0,1] \to U$ such that $\gamma(0) = p$
        and $\gamma(1) = q$.
        Let
        \[ T = \left\{ t \in [0,1] \, | \, \ran{\gamma\big|_{[0, t]}} 
            \subseteq E_1 \right\}.
        \]

        Since $0 \in T$, $T \ne \emptyset$.  
        This means $T$ has a supremum; let it be $t_0$.
        Since $\gamma$ is continuous, define $z_0 = \gamma(t_0) \in U$.
        We split the next part into two cases each with two
        subcases.
        \begin{enumerate}
            \item $z_0 \in E_1$.  There are two cases to consider.
                \begin{enumerate}
                    \item Suppose $t_0 = 1$.
                        Note that $\gamma(t_0) = \gamma(1) = q$.
                        This means $q \in E_1$, a contradiction.

                    \item Suppose $t_0 < 1$.
                        We can extend the supremum because $t_0$
                        is less than $1$.
                \end{enumerate}

            \item $z_0 \in E_2$
                This is left as an exercise to the reader.
        \end{enumerate}

        The lemma completes the theorem, so one of $E_1$
        or $E_2$ is empty.

    \end{proof}

    \begin{exercise}
        Prove case $2$ of the lemma in the previous theorem.
    \end{exercise}
    

\end{proof}

\end{document}
