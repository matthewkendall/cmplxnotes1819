\section{Nov. 9, 2018}
\subsection{Cauchy-Riemann equations}

\begin{theorem}[Cauchy-Riemann]
    If $f$ is analytic at $z$, then the \term{Cauchy-Riemann equation} holds:
    \begin{align*}
        \frac{\partial f}{\partial y} = i \frac{\partial f}{\partial x}.
    \end{align*}
\end{theorem}

\begin{proof}
    Let $z = x + iy$.
    Since $f$ is analytic at $z$, let's first take $\frac{\partial f}{\partial x}$:
    \begin{align*}
        \frac{\partial f}{\partial x} &= \lim_{r \to 0} = f'(z).
    \end{align*}
    We also know:
    \[ \frac{\partial f}{\partial y} = \lim_{r \to 0} \frac{f(z+ir) - f(z)}{r} = i \lim_{r \to 0} \frac{f(z+ir) - f(z)}{ir} = if'(z). \]
    Therefore,
    \begin{align*}
        \frac{\partial f}{\partial y} = i \frac{\partial f}{\partial x}.
    \end{align*}
        
\end{proof}
\noindent
We also write the Cauchy-Riemann equations in their real forms:
\begin{align*}
    \frac{\partial u}{\partial x} &= \frac{\partial v}{\partial y}, \\
    \frac{\partial u}{\partial y} &= -\frac{\partial v}{\partial x}.
\end{align*}
Here is a proof of why we can do this.

\begin{proof}
    Let $z = x + iy$ and $f(z) = f(x,y) = u(x,y) + iv(x,y)$.
    Let's first take $\frac{\partial f}{\partial x}$:
    \begin{align*}
        \frac{\partial f}{\partial x} &= \lim_{r \to 0} \frac{f( (x+r) + iy) - f(x+iy)}{r} \\
        &= \lim_{r \to 0} \frac{u(x+r,y) + iv(x+r,y) - (u(x,y) + iv(x,y))}{r} \\
        &= \lim_{r \to 0} \frac{u(x+r,y) - u(x,y)}{r} + i\lim_{r \to 0} \frac{v(x+r,y) - v(x,y)}{r} \\
        &= \frac{\partial u}{\partial x} + i \frac{\partial v}{\partial x}.
    \end{align*}
    Similarly,
    \[ \frac{\partial f}{\partial y} = \frac{\partial u}{\partial y} + i\frac{\partial v}{\partial y}. \]
    Equating real and imaginary parts gives us:
    \begin{align*}
        \frac{\partial u}{\partial x} &= \frac{\partial v}{\partial y}, \\
        \frac{\partial u}{\partial y} &= -\frac{\partial v}{\partial x}.
    \end{align*}

\end{proof}
\noindent
If two real valued functions $u(x,y)$ and $v(x,y)$ satisfy the real form of the Cauchy-Riemann equations, they are called \term{harmonic conjugates}.

\begin{exercise}
    Find all harmonic conjugates of $u(x,y) = x^2 - y^2$.
\end{exercise}

\noindent
Now for the converse.
\begin{theorem}
    Suppose $f$ satsifies the Cauchy-Riemann equations at $z$ and has continuous partial derivatives $\frac{\partial f}{\partial x}$ and $\frac{\partial f}{\partial y}$.  Then $f$ is analytic at $z$, and $f'(z) = \frac{\partial f}{\partial x}(z)$.
\end{theorem}

\begin{example}
    Take $f(z) = e^z$, with the domain of $f$ being $\C$.
\end{example}

\begin{proof}
    Let $h = r+is$.

    We expand out the numerator of the difference quotient:
    \[ f(z+h) - f(z) = f(z + (r+is)) - f(z+r) + f(z+r) - f(z). \]
    Note that by the Mean Value Theorem:
    \begin{align*}
        f(z + (r+is)) - f(z+r) &= \frac{\partial f}{\partial y} (z + r + i\phi s) s \\
        f(z + r) - f(z) &= \frac{\partial f}{\partial x} (z + \theta r) r,
    \end{align*}
    for some $0 < \phi < 1$ and $0 < \theta < 1$.
    Let
    \begin{align*}
        \epsilon_1 (h) &= \frac{\partial f}{\partial y} (z + r + i\phi s) - \frac{\partial f}{\partial y}(z) \\
        \epsilon_2 (h) &= \frac{\partial f}{\partial x} (z + \theta r) - \frac{\partial f}{\partial x} (z)
    \end{align*}
    Our numerator therefore becomes:
    \begin{align*}
        f(z+h) - f(z) &= f(z + (r+is)) - f(z+r) + f(z+r) - f(z) \\
        &= \frac{\partial f}{\partial y} (z + r + i\phi s) s + \frac{\partial f}{\partial x} (z + \theta r) r \\
        &= \left( \frac{\partial f}{\partial y}(z) + \epsilon_1 (h) \right)s + \left(\frac{\partial f}{\partial y}(z) + \epsilon_2 (h) \right)r \\
        &= \left( i\frac{\partial f}{\partial x} (z) + \epsilon_1(h) \right) s + 
        \left( \frac{\partial f}{\partial x} (z) + \epsilon_2(h) \right ) r \tag{By Cauchy-Riemann} \\
        &= h \frac{\partial f}{\partial x} (z) + s \epsilon_1(h) + r \epsilon_2(h).
     \end{align*}

     \noindent
     Finally we can show our difference quotient approaches $0$:
     \begin{align*}
        \left| \frac{f(z+h) - f(z)}{h} - \frac{\partial f}{\partial x} (z) \right|
        &= \left| \frac{s \epsilon_1(h) + r \epsilon_2 (h)}{h} \right| \\
        &\le \frac{|s|}{|h|} |\epsilon_1(h)| + \frac{|r|}{|h|} |\epsilon_2(h)| \tag{Triangle Inequality} \\
        &\le |\epsilon_1(h)| + |\epsilon_2(h)|
     \end{align*}
     Note that continuity of partials of $f$ imply $\epsilon_1(h)$ and $\epsilon_2(h)$ both approach $0$.
     This proves our theorem.
\end{proof} 

\begin{exercise}
    Prove $f(z) = {\Log}(z)$, the principal log of $z$, is analytic on $z \in \C \setminus (\R^- \cup \{0\})$.
\end{exercise}

\subsection{Trigonometric and Hyperbolic Functions}

\begin{definition}
    A function $f$ is called \term{entire} if it is analytic in all of $\C$.
\end{definition}

We define
\begin{align*}
    \cos (z) &:= \frac{e^{iz} + e^{-iz}}{2} \\
    \sin (z) &:= \frac{e^{iz} - e^{-iz}}{2i} \\
    \tan (z) &:= \frac{\sin z}{\cos z}
\end{align*}


\noindent 
We note that $\cos z = 0$ if $z \in \R$.  This is because when we set the numerator of $\cos$ to $0$, we find
that $e^{iz} = -e^{-iz}$. 
This means $e^{2iz} = -1$, which happens when $2iz = i(\pi + 2\pi k)$ for some integer $k$.
We then find $z = \pi/2 + \pi k$, as expected.


\noindent
We define the secant, cotangent, and cosecant as the reciprocals of cosine, tangent, and sine respectively.

\noindent
Periods of trigonometric functions must be real because if they were complex, we can start with a real root and jump to a complex one, which as we know is not possible.

\noindent
We can verify that $\cos (-z) = \cos z$ and $\sin (-z) = -\sin z$ by plugging in to our definitions.

\begin{definition}
    Define the \term{multivalued inverse tangent} and 
    \term{principal inverse tangent} as:
    \begin{align*}
        \arctan w &= \frac{1}{2i} \log \left( \frac{1 + iw}{1 - iw}\right), \\
        \Arctan w &= \frac{1}{2i} \Log \left( \frac{1 + iw}{1 - iw}\right).
    \end{align*}
\end{definition}
We see that $\dom (\Arctan) = \C \setminus \{i, -i\}$.
Also,
\begin{align*}
    \tan( \arctan w ) &= w, \\
    \arctan( \tan z ) &= z + \pi \Z.
\end{align*}

\begin{definition}
    Let the \term{hyperbolic sine}, \term{hyperbolic cosine},
    and \term{hyperbolic tangent}
    be defined as:
    \begin{align*}
        \sinh z &= \frac{e^z - e^{-z}}{2}, \\
        \cosh z &= \frac{e^z + e^{-z}}{2}, \\
        \tanh z &= \frac{e^z - e^{-z}}{e^z + e^{-z}}
        = \frac{e^{2z} - 1}{e^{2z} + 1}.
    \end{align*}
\end{definition}

There are a bunch of properties you can check with hyperbolic
sine and cosine, such as parity (what happens when you 
replace $y$ with $-y$), addition.

\subsection{\texorpdfstring{$f \equiv c$}{f identically constant} conditions}

Let $f$ be analytic throughout a \term{region} $U$: an oppen and connected set in $\C$.
$f$ reduces to a constant function when:
\begin{enumerate}

\item $f'(z) \equiv 0$ on $U$.

\item $\real f \equiv c \in \C$.

\item If $|f| \equiv c \in \C$.

\item If ${\Arg} f \equiv c \in \C$.

\end{enumerate}

\noindent
We leave all of these properties as exercises to the reader.
They are all applications of the Cauchy-Riemann equations.

\begin{exercise}
    Prove the $f \equiv c$ conditions.
\end{exercise}

