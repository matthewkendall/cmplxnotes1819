\section{April 8, 2019}

\subsection{Chains}
Let $\mathscr{P}$ be the set of all piecewise smooth paths.
Treat \term{formal sums} of paths as component-wise addition of paths: if 
\[ A = \sum_{\gamma \in \mathscr{P}} a_{\gamma} [\gamma] 
\;\;\;\; \text{and} \;\;\;\; B = \sum_{\gamma \in \mathscr{P}} b_{\gamma} [\gamma], \]
then
\[ A+B := \sum_{\gamma \in \mathscr{P}} (a_{\gamma} + b_{\gamma})
[\gamma]. \]
The \textit{zero element} is:
\[ 0 := \sum_{\gamma \in \mathscr{P}} 0[\gamma]. \]
The inverse of a formal sum $A$ is:
\[ -A := \sum_{\gamma \in \mathscr{P}} (-a_{\gamma}) [\gamma]. \]

\begin{definition}
    Define the following equivalence relations ${\sim}_{a}$, ${\sim}_{b}$, and ${\sim}_{c}$ on formal sums of paths
    as follows:
    \begin{itemize}
        \item ${\sim}_{a}$ states that 
            a concatenation of paths $\gamma$ and $\delta$ is equivalent
            to the sum of the paths $\gamma$ and $\delta$. %rephrase please
        \item ${\sim}_{b}$ states that the reversal
            of a path $\gamma$ is equivalent to the additive inverse of $\gamma$.
        \item ${\sim}_{c}$ states that if a path
            $\gamma$ is reparameterized into a path $\gamma 
            \circ \phi$, where $\phi$ is some smooth and increasing
            function, then $[\gamma] {\sim}_{c} 
            [\gamma \circ \phi]$. 
    \end{itemize}
\end{definition}

It can be easily checked that $\sim_a$, $\sim_b $ and $\sim_c$
satisfy the necessary equivalence relation conditions.

\begin{definition}
    Define the equivalence relation $\sim$ on formal sums of paths
    as follows:
    $A \sim B$ if there exists a finite sequence of formal sums
    $(A, C_1, \ldots, C_n, B)$ such that
    \begin{itemize}
        \item $A \sim_a C_1$ or $A \sim_b C_1$ or $A \sim_c C_1$.
        \item $C_1 \sim_a C_2$ or $C_1 \sim_b C_2$ or $C_1 \sim_c C_2$, and so on, until:
        \item $C_n \sim_a B$ or $C_n \sim_b B$ or $C_n \sim_c B$.
    \end{itemize}
   
\end{definition}

\begin{definition}
    A \term{chain} is an \textit{equivalence class}
    of \term{formal sums}
    of paths:
    \[ A = \sum_{\gamma \in \mathscr{P}} a_{\gamma} [\gamma]
        = a_1[\gamma_1] + a_2[\gamma_2] + \cdots + a_n[\gamma_n].
    \]
\end{definition}

\begin{theorem}
    If $A \sim A'$ and $B \sim B'$, then $A + B \sim A' + B'$.
\end{theorem}

\begin{proof}
    The details are a little messy, but the idea is to show
    \[ A + B \sim A' + B \sim A' + B'. \]
\end{proof}

Let's introduce the shorthand notation for integrating over formal sum
$A = a_1[\gamma_1] + a_2[\gamma_2] + \cdots + a_n[\gamma_n] $
as
\[ \int_A f := \sum_{j=1}^n \int_{\gamma_j} a_j [\gamma_j]. \]
If some of the $\ran \gamma_j \nsubseteq \dom f$, then 
\[ \int_A := \int_B f \]
where $B$ is some formal sum and $A \sim B$ and $B$ only involves
paths inside $\dom f$.
If there exists no such $b$, then $\int_A f$ is not defined.

\begin{theorem}
    If $A \sim B$, then
    \[ \int_A f = \int_B f. \]
\end{theorem}

\begin{proof}
    We examine the rules of chain equivalence:
    \begin{itemize}
        \item \textit{Concatenation:} We know the integral over
            a concatenation of paths is equal to the sum of the
            integrals over each path.
        \item \textit{Reversal:} We know the integral over the
            reversal of a path $\gamma$ is equal to the negative
            of the integral over path $\gamma$.
        \item \textit{Reparameterization:} We know that
            the integral over a path is independent of the 
            parameterization.
    \end{itemize}
    By the definition of $\sim$, we know that $A$ and $B$
    are equivalent after a series of $\sim_a$, $\sim_b$, and
    $\sim_c$ equivalences.
    Since each of these equivalences preserve the integral,
    \[ \int_A f = \int_B f. \]
\end{proof}

\begin{definition}[Integration on a Chain]
    If $\alpha$ is a chain in $\dom f$, $\alpha = {[A]}_{\sim}$,
    define
    \[ \int_{\alpha} f := \int_A f, \]
    where $A \in \alpha$.
\end{definition}

\subsection{Cycles}
\begin{definition}
    A \term{cycle} is the equivalence class of a sum of loops:
    \[ \alpha = {[ [\gamma_1] + \cdots + [\gamma_n]]}_{\sim}, \]
    where $\gamma_1, \gamma_2, \ldots, \gamma_n$ are loops.
\end{definition}

\begin{theorem}
    Consider a formal sum of paths $A = [\gamma_1] + \cdots + 
    [\gamma_n]$.
    Then $\alpha = {[A]}_{\sim}$ is a cycle iff the multiset\footnote{A set with repeated elements} of initial points is equal
    to the multiset of endpoints.\footnote{Two multisets are 
    equal if they have the same number of each element.}
\end{theorem}

\begin{proof}
    Apply old Cauchy Goursat to a sum of cycles.
\end{proof}
 
\begin{theorem}[Cauchy Goursat for Cycles in a Disk]
    If $f$ is analytic in an open disk $\Delta$ and if $\beta$
    is a cycle in $\Delta$, then
    \[ \int_{\beta} f(z) \; dz = 0. \]
\end{theorem}
A cycle $\beta$ that avoids a point $p \in \C$ has a 
\textit{winding number} with respect to $p$:
\[ n(\beta, p) = \frac{1}{2\pi i}\int_{\beta} \frac{dw}{w-p}.\]
If we write $\beta = [\gamma_1] + \cdots + [\gamma_n]$, then
\[ n(\beta, p) = \sum_{j=1}^n n(\gamma_j, p). \]

\begin{definition}
    A region $U$ is called \term{simply connected} if every point
    $z_0 \in \C \setminus U$ can be joined to $\infty$ by a
    continuous curve avoiding $U$.
\end{definition}

Another way to think about a simply connected region is that
it has no holes.

\subsection{Cauchy Goursat in a Simply Connected Region}
\begin{theorem}
    Let $f$ be analytic in a simply connected region $U$.
    Let $\beta$ be a cycle inside $U$.
    Then 
    \[ \oint_{\beta} f = 0. \]
\end{theorem}

\begin{proof}
    We will prove the existence of an antiderivative of $f$.
    If $f$ has an antiderivative, then it must be independent
    of path.
    Since $\beta$ is a sum of loops, $f$ integrated over $\beta$
    will be $0$.

    \noindent
    Let $p \in \C$ be a fixed point in the plane and 
    \[ F(z) := \int_{\sigma_{p, z}} f(w) \; dw, \]
    where $\sigma_{p, z}$ is an arbitrary zig-zag path from
    $p$ to $z$ staying inside $U$.
    The first and most difficult step is to show $F$ is well
    defined: that is, for any two zig-zag paths $\sigma_1$
    and $\sigma_2$ going from $p$ to $z$:
    \[ \int_{\sigma_1} f = \int_{\sigma_2} f. \]
    Let $\gamma = [\sigma_1] - [\sigma_2]$ be a loop.
    It remains to show
    \[ \int_{\gamma} f = 0. \]
    Let $G$ be a grid obtained by extending all segments of $\sigma_1$ 
    and $\sigma_2$ into lines.
    Denote the bounded rectangles by $R_j$ with centers $a_j$
    and the unbounded rectanges by $R_k'$ with centers $a_k'$.

    If there are no bounded rectangles, then $\sigma_1$ and
    $\sigma_2$ are retracings of one another, so by the
    additivity integration, $\int_{\sigma_1} f = \int_{\sigma_2}
    f$.

    Otherwise, suppose the bounded rectangles are $R_1, R_2, 
    \ldots, R_N$.
    Let
    \[ \gamma_0 = \sum_{j=1}^N n(\gamma, a_j) \Gamma(R_j), \]
    where $\Gamma(R_j)$ is a counterclockwise loop around
    rectangle $R_j$ starting and ending at the bottom left 
    corner.
    \begin{claim*}
        \[\gamma \sim \gamma_0. \]
    \end{claim*}
    \begin{proof}
        Abbreviate $n_j = n(\gamma, a_j)$.
        Note that $n(\gamma, a_j) = n(\gamma_0, a_j)$ for
        any $j = 1, \ldots, N$ because
        \[ n(\gamma_0, a_j) = n\left( \sum_{k=1}^N n_k \Gamma(R_k),
            a_j \right)
        = \sum_{k=1}^N n_k n(\Gamma(R_k), a_j)
        = n_j = n(\gamma, a_j). \]
        Let $\mu = [\gamma] - [\gamma_0]$.
        Similarly, $n(\gamma, a_j') = n(\gamma_0, a_j')$:
        \[ n(\gamma_0, a_j') = n\left( \sum_{k=1}^N n_k \Gamma(R_k),
            a_j' \right)
        = \sum_{k=1}^N n_k n(\Gamma(R_k), a_j')
        = 0 = n(\gamma, a_j'). \]
        We used the fact that $n(\Gamma(R_j), a_k) = \delta_{jk}$ and
        $n(\Gamma(R_k), a_j') = 0$ because $a_j'$
        is outside all of the bounded rectangles.
        Now let 
        \[ \mu = [\gamma] - [\gamma_0]. \]
        From above we know $n(\mu, a_j) = n(\mu, a_k') = 0$,
        so we wish to show $\mu \sim 0$.

        Consider two adjacent rectangles $R_j$ and $R_k$.
        Let $\sigma_{jk}$ be a directed line segment 
        that is the shared edge between $R_j$ and $R_k$
        ($R_j$ is to the left of $R_k$) and
        is oriented in the same direction as in $\Gamma(R_j)$.
        Let $\sigma_{jk}'$ be defined similarly, except
        $R_k$ is now an unbounded rectangle $R_k'$.
        Say $\mu$ has a simplest form expression as a formal
        sum of $[\sigma_{jk}]$ and $[\sigma_{kl}']$ terms and
        suppose $c[\sigma_{jk}]$ appears in $\mu$.
        We wish to show $c=0$.
        
        Consider
        \[ \delta = [\mu] - c[\Gamma(R_j)]. \]
        We claim $n(\delta, a_j) = n(\delta, a_k)$.
        This is because the $c[\sigma_{jk}]$ term is canceled
        in the $[\mu]$ and $c[\Gamma(R_j)]$ by construction,
        so $a_j$ and $a_k$ can be connected by a line segment
        that does not intersect $\delta$.
        On the one hand, 
        \[ n(\delta, a_j) = 0 - c = -c. \]
        But
        \[ n(\delta, a_k) = 0 - 0 = 0. \]
        Therefore, $-c = 0$ so $c = 0$.

        This means every term of $\mu$ has a $0$ coefficient
        in its simplest form, which means $\mu$ is chain 
        equivalent to $0$.
        This is equivalent to our desired claim.
    \end{proof}
    \begin{claim*}
        \[ \int_{\gamma_0} f = 0. \]
    \end{claim*}
    \begin{proof}
        Using the definition of $[\gamma_0]$:
        \[ \int_{\gamma_0} f = \sum_{j=1}^N n_j \int_{\Gamma(R_j)}
        f. \]
        If $\Gamma(R_j)$ lies entirely within $R_j$, then
        \[ \int_{\Gamma(R_j)} f = 0 \]
        by our first form of Cauchy Goursat.
        Otherwise, $a_j$ can be connected to a point $a$
        in the complement of $U$, and $n(\gamma, a_j) = n(\gamma
        , a) = 0$ because $U$ is simply connected.
        If $a$ is on the boundary of $\Gamma(R_j)$, note that
        it cannot be on the same side as the loop $\gamma$
        by construction.
        Therefore,
        \[ \int_{\gamma_0} f = 0. \]
    \end{proof}
    Working our way upwards, because $\gamma \sim \gamma_0$,
    \[ \int_{\gamma} f = \int_{\gamma_0} f = 0. \]
    By our construction of $\gamma$, this means
    \[ \int_{\sigma_1} f = \int_{\sigma_2} f, \]
    which means $F$ is well defined.
    $F$ is indeed an antiderivative of $F$, verified
    in the Independence of Path section.
    This completes the proof, so
    \[ \oint_{\beta} f = 0. \]
\end{proof}

Note that the condition of $U$ being simply connected can
be weakened into $n(\beta, a) = 0$ for all $a$ in the complement
of $U$.

