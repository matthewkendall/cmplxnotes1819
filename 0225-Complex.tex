\documentclass[notes]{subfile}

\begin{document}

\section{Feb. 24, 2019}

$f$ analytic in a region $U$.
Take $a \in U$ and $r > 0$ such that $\overline{D}_r(a) \subseteq U$.  Let $C$ be the circle of radius $r$ centered at $a$.
Choose $z \in \overline{D}_r(a)$, then
\[ f(z) = \underbrace{\sum_{k=0}^{n-1} 
        \frac{f^{(k)}(a)}{k!}
\cdot (z-a)^k }_{T_{n-1}(z)} + \underbrace{\frac{(z-a)^n}{2\pi i} \oint_C \frac{f(w)}{(w-a)^n(w-z)} \; dw}_{R_n(z)}. \]

We call $T_{n-1}(z)$ the degree $n-1$ \term{Taylor Polynomial}
of $f$ and we call $R_n(z)$ the \term{remainder term}
of $f$.

\begin{lemma}
    $R_n (z) \to 0 $ as $n \to \infty$.
    \label{}
\end{lemma}

Sidenote: If $R$ is the \term{radius of convergence}, then $R \ge r$.

\begin{proof}
    
    Let's estimate the remainder term.

    \begin{align*}
        |R_n(z)| &\le r M_{a, r}(f) \frac{1}{r^n(r - |z-a|)} |z-a|^n 
        \tag{ML Theorem} \\
        &= \frac{rM_{a, r}(f)}{r - |z-a|} \frac{|z-a|^n}{r^n} \to 0
    \end{align*}
    as $n \to \infty$.
\end{proof}


Note that
\[ \lim_{n \to \infty} T_{n-1}(z)  = f(z). \]

Let's now prove the remainder term.

\begin{proof}
    Write
    \begin{align*}
        f_n(w) &= \frac{f(w)}{(w-a)^n} - \frac{f(a)}{(w-a)^n}
        - \cdots - \frac{f^{(n-1)}(a)}{(n-1)!(w-a)} \\
        f_n(z) &= \frac{1}{2\pi i} \oint_C \frac{f_n(w)}{(w-z)} \; dw \\
        &= \frac{1}{2\pi i} \oint_C \frac{f(w) \;dw}{(w-a)^n(w-z)} - 
        \frac{f(a)}{2\pi i}\oint_C \frac{dw}{(w-a)^n(w-z)}
        - \cdots
    \end{align*}

    Let 
    \[ F(v) = \frac{1}{2\pi i} \oint_C \frac{dw}{(w-v)(w-z)}
    = \frac{1}{2\pi i} \oint_C \frac{1}{v-z} \left( \frac{1}{w-z} - \frac{1}{z-v}\right) \; dw
    = \frac{1}{z-v} \left( n(C, z) - n(C, v)\right) \equiv 0.
    \]
    The integral in the second to last term is $F'(v)$, and 
    the next to last integral is $F''(v)$ and so on.
    Therefore, all the terms cancel except for the first term,
    so we're done.
\end{proof}

\end{document}
