\documentclass[notes]{subfile}

\begin{document}

\section{Dec. 12, 2018}

\subsection{Independence of Path}

We say $f(z)$ is \term{independent of path} if
\[ \int_{\gamma} f(z) \; dz = G(\gamma(0), \gamma(N)) \]
for some function $G$, where $[0,N] = \dom{\gamma}$ for 
\textit{all} paths $\gamma$ in region $U$.

\begin{theorem}[Independence of Path - IOP]
    Suppose $f(z)$ is continuous and has an antiderivative $F(z)$ throughout the region $U$.
    Let $F(z)$ be the antiderivative of $f$.
    Then
    \[ \int_{\gamma} f(z) \; dz = F(\gamma(N)) - F(\gamma(0)). \]
\end{theorem}

\begin{proof}
    The proof is another string of integrals.
    \begin{align*}
        \int_{\gamma} f(z) \; dz &= \int_0^N f(\gamma(t)) \gamma'(t) \; dt \\
        &= \int_0^N F'(\gamma(t)) \gamma'(t) \; dt \\
        &= \int_0^N \frac{d}{dt} (F(\gamma(t))) \; dt \\
        &= F(\gamma(N)) - F(\gamma(0)).
    \end{align*}
    This completes the proof.
\end{proof}

Here is the Fundamental Theorem of Complex Calculus.
\begin{theorem}[Fundamental Theorem of Complex Calculus - FTCC]
    Given a function $F$ such that $F$ is continuous on an open
    set $U$ containing $\ran{\gamma}$, then
    \[ \int_{\gamma} F'(z) \; dz = F(\gamma_{\text{term}}) -
    F(\gamma_{\text{init}}). \]
\end{theorem}

Now for the converse of IOP.
\begin{theorem}[Converse of IOP]
    We prove if $\int_{\gamma} f(z) \; dz$ IOP, then there exists
    an analytic function $F$ on $U$ such that $F'(z) = f(z)$ for
    all $z \in U$.
\end{theorem}

\begin{proof}
    Prove this later.
\end{proof}

\subsection{Cauchy-Goursat Theorem}
Let 
\[ I = \int_{\gamma} (z-a)^n \; dz. \]
If $n \in \Z$ and $\gamma$  is a piecewise smooth loop, then
\begin{enumerate}
    \item For $n \ge 0$, $I = 0$ for any $a \in \C$.
    \item For $n \le -2$, provided $a \notin \ran{\gamma}$,
        $I = 0$.
    \item If $n = -1$,
        \[ I = \begin{cases}
                0 & \text{if $\ran{\gamma} \subseteq \C \setminus \R^-$.} \\
                2\pi i & \text{if $\gamma$ is a counterclockwise circle 
                centered at $a$.}
            \end{cases}
        \]
       
\end{enumerate}

\begin{theorem}[Cauchy-Goursat I]
    Let $f(z)$ be a function analytic on the interior of 
    $R = [a,b] \times [c,d]$.
    Let $\Gamma(R)$ be the boundary of the rectangle traversed 
    counterclockwise starting from the bottom let corner.
    Then
    \[ \oint_{\Gamma(R)} f(z) \; dz = 0. \]
\end{theorem}

\begin{proof}
    \textbf{paste here later}
\end{proof}

\begin{theorem} [Cauchy-Goursat II]
    Let $f$ be analytic in region $U$ containing a rectangle
    $R = [a,b] \times [c,d]$ except at finitely many points
    $p_1, p_2, \ldots, p_n$ such that for $1 \le i \le n$:
    \[ \lim_{z \to p_j} (z-p_j) f(z) = 0.\]
    Then
    \[ \oint_{\Gamma(R)} f(z) \; dz = 0. \]
\end{theorem}

\begin{proof}
    \textbf{paste here later}
\end{proof}

\begin{theorem}[Cauchy-Goursat III]
    Let $f$ be analytic in an open disk $\Delta = D_r(p)$
    except at finitely many points $p_1, p_2, \ldots, p_n$.
    Let $\gamma$ be \textit{any} loop in $\Delta \setminus 
    \{ p_1, p_2, \ldots, p_n \}$ such that for all of these points,
    \[ \lim_{z \to p_j} (z-p_j)f(z) = 0.\]
    Then
    \[ \oint_{\gamma} f(z) \; dz = 0. \]
\end{theorem}

\begin{proof}
    \textbf{paste here later}
\end{proof}



\end{document}
