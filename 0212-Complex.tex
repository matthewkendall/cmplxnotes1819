\documentclass[notes]{subfile}

\begin{document}

\section{Feb. 12, 2019}

\subsection{Finite Taylor Development}

First we talk about \term{removable singularities}.
We suppose $f$ is analytic in $U \setminus \{a\}$,
where $U$ is a region (open and connected set).

\begin{theorem}
    There exists an analytic function $g$ in $U$ agreeing with $f$
    on $U \setminus \{ a \}$ if and only if $f$ is not too bad
    at $a$:
    \[ \lim_{z \to a} (z-a)f(z) = 0. \]
\end{theorem}

Typically, $g$ is called $f$ again.

\begin{proof}
    Take a circle $C$ centered at $a$ with radius $r > 0$ such that
    $\overline{D_r}(a) \subseteq U$.

    Define
    \[ Z_a = \frac{1}{2\pi i} \oint_C \frac{f(a)}{a-z} \; dw. \]
    Now let 
    \[ g(z) = \begin{cases}
            f(z) & \text{when $z \ne a$} \\
            Z_a & \text{when $z=a$}
        \end{cases}
    \]

    We show $g$ is analytic, but since it agrees with $f$
    everywhere except $a$, we just need to show $g$ is analytic
    at $a$.

    We write out a difference quotient for $g$:
    \begin{align*}
        \frac{g(a+h) - g(a)}{h} &= \frac{f(a+h) - Z_a}{h} \\
        &= \frac{1}{2\pi i} \oint_C f(w) \cdot \frac{1}{h}
        \left[ \frac{1}{w-a-h} - \frac{1}{w-a} \right] \; dw \\
        &= \frac{1}{2\pi i} \oint_C f(w) \frac{1}{(w-a-h)(w-a)} \; dw.
    \end{align*}

    We suspect we can float the $h$ through the integral so the 
    difference quoetient approaches
    \[ W_a = \frac{1}{2\pi i} \oint_C f(w) \frac{1}{(w-a)^2} \; dw. \]

    By the ML theorem:
    \[ \left[ \frac{g(a+h) - g(a)}{h} - W_a  \right] \le
        \frac{1}{2\pi} \cdot 2\pi r \cdot M_{a,r}(f) \cdot
        \max_{w \in C} \left| \frac{1}{(w-a)(w-a-h)} - 
    \frac{1}{(w-a)^2} \right| \]
    But as we know, as $ h \to 0$:
    \[ 
        \max_{w \in C} \left| \frac{1}{(w-a)(w-a-h)} - 
    \frac{1}{(w-a)^2} \right| = 
    \max_{w \in C} \left| \frac{h}{(w-a)^2(w-a-h)} \right| \to 0\]
    Therefore, $g$ is analytic at $a$ and thus analytic in $U$.

    \smallbreak
    Conversely, if such a $g$ exists and and $f$ is defined
    on $U \setminus \{ a \}$:
    \[ \lim_{z \to a} f(z)(z-a) = \lim_{z \to a} g(z)(z-a)
    = g(a)(a-a) = 0,\]
    where the second to last equation follows from the continuity of $g$.


\end{proof}

A good question is: where is the limit condition used in the proof
of the existence of $g$?

\smallbreak
\noindent
If a function $f$ is analytic in a region $U$, for a point
$a \in U$, define
\[ f_1(z) = \frac{f(z)-f(a)}{z-a} \text{  when $z \in U \setminus \{a\}$. } \]

We also notice that $f$ is not too bad at $a$:
\[ \lim_{z \to a} f_1(a) (z-a) = \lim_{z\to a} f(z) - f(a) = 0. \]
This means we can define 
\[ f_1(a) = \oint_C \frac{f_1(w)}{(w-a)^2} \; dw, \]
where $C$ is a circle of radius $r$ centered at $a$ such that the
disk $D_r(a)$ is contained inside $U$. 
Note that $f_1$ exists because $f_1(w)/(w-a)^2$ is a continuous
integrand because $w \in C$, meaning the loop integral exists.

This means we can extend $f_1$ so that the function is analytic
in $U$.
Now we write
\[ f(z) = f(a) + (z-a)f_1(z). \]
We can iterate $n$ times to get:
\[ f(z) = f(a) + f_1(a) (z-a) + f_2(a) (z-a)^2 + \cdots +
f_{n-1}(a)(z-a)^{n-1} + (z-a)^nf_n(z). \]
We can compute each coefficient $f_k(a)$ by differentiating both sides of the previous equation $k-1$ times:
\[f^{(k)}(a) = k!(a-a)^0f_k(a) = k!f_k(a). \]
This means
\[ f_k(a) = \frac{f^{(k)}(a)}{k!}.\]
for all $k \ge 0$.  Now we have a Taylor series expansion of $f(z)$.
\begin{theorem}
    \[ f(z) = \left( \sum_{k=0}^{n-1} \frac{f^{(k)}(a)}{k!} (z-a)^k \right) + (z-a)^nf_n(z). \]
\end{theorem}

\noindent
It turns out we can also find an explicit form for $f_n(z)$ in terms
of $f$.

\begin{exercise}
    Let $C$ be a circle of radius $r$ centered at $a$ such that
    $\overline{D_r}(a) \subseteq U$.
    Using the Taylor Series expansion, prove that
    \[ f_n(z) = \frac{1}{2\pi i} \oint_{C} \frac{f(w) \; }{(w-a)^n (w-z)}
    \; dw. \]
\end{exercise}

Let 
\begin{align*}
    T_{n-1}(z) &= f(a) + f'(a)(z-a) + \cdots + (z-a)^{n-1}
    \frac{f^{(n-a)}(a)}{(n-1)!}. \\
    R_n(z) &= (z-a)^n \frac{1}{2\pi i}\oint_C \frac{f(w) \; }{(w-a)^n (w-z)}
    \; dw.
\end{align*}

We can bound the remainder term $|R_n(z)|$ with the ML theorem:
\[ |R_n(z)| \le \frac{|z-a|^n}{2\pi}  
    \cdot 2\pi r \frac{1}{r^n} \cdot \frac{1}{r - |z-a|}
    \cdot M_{a,r}(f) = \frac{r M_{a,r}(f)}{r - |z-a|} \cdot 
\left( \frac{|z-a|}{r} \right)^n. \]

Note that $| |z-a|/r | < 1$ so as $n \to \infty$, $|R_n(z)| \to 0$.
This means $\lim_{z \to \infty} T_{n} (z) = f(z)$.

\noindent
This leads to the following theorem.
\begin{theorem}
    If $f$ is analytic in region $U$ and if $a \in U$, then $f$
    can be represented by its Taylor series in at least the largest
    disk centered at $a$ inside $U$.
\end{theorem}
The radius of the largest disk for which the Taylor Series 
matches the function is called the \term{radius of convergence}.

\begin{exercise}
    Let $f(z) = 1/(1-z)$.  Show that the radius 
    of convergence centered at $a=3$ is $2$.
\end{exercise}


\end{document}
